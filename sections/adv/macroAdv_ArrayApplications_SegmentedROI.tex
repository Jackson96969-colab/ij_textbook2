% Array application to measure segmented ROI. 

\subsubsection{Acquiring intensity profile from segmented line ROI}
 
In recent version of ImageJ, selection thickness controls the width of segmented
line ROI when you do \ijmenu{[Analyze > Plot Profile])}. We try to mimick this
behavior in macro, and instead of choosing the line ROI thickness using GUI, the
macro asks the user to input the thickness. 

In the code below, there is only one macro. Two functions are added at the
bottom. One is for profile plotting and the last one is for listing
intensity profile data in the result table. Strategy of this macro is to use
straight line selection for each segment, measure that segment and then profiles
are concatenated to the total profile array.

%\lstinputlisting[morekeywords={*, newArray, selectionType, getProfile, setResult, updateResults}]{code/code20_75.ijm}
%\lstinputlisting[morekeywords={*, getSelectionCoordinates, makeLine, Plot,
% create, setLimits, setColor, add, show}]{code/code20_75.ijm}
\lstinputlisting[morekeywords={*, getSelectionCoordinates, makeLine, Plot,
create, setLimits, setColor, add, show, Array, concat,
getStatistics}]{code/code20_76.ijm}

\begin{itemize}
\item Lines 2 - 16: Main part, macro for the segmented line ROI measurement.  

\item Line 3: Check if the selection type is a segmented line ROI. If not, macro
terminates leaving a message.

\item Line 4: Reads the x and y coordinates of the segmented line
and store them in two arrays \ilcom{xCA} and \ilcom{yCA}.

\begin{indentCom}
\textbf{getSelectionCoordinates(xCoordinates, yCoordinates)}\\
Returns two arrays containing the X and Y coordinates of the points that define the current selection. 
\end{indentCom}

\item Line 5 - 7: Asks the user to input width of the segmented ROI. The ROI
line width is set to that value.

\item Line 8: A new array \ilcom{totalprofile} is created, initialized without
any element. This new array will store the profile data of full ROI.

\item Line 9 - 13: Profile measurement by placing straight line ROI,
for wach segment of the original ROI. \ilcom{makeLine} function is used for this
purpose, and \ilcom{getProfile} returns intensity profile of the corresponding
line ROI. Profile data in \ilcom{thisprofile} array are concatenated to
\ilcom{totalprofile} array using \ilcom{Array.concat}.

\begin{indentCom}
\textbf{makeLine(x1, y1, x2, y2)}\\
Creates a new straight line selection. The origin (0,0) is assumed to be the upper left corner of the image. Coordinates are in pixels. With ImageJ 1.35b and letter, you can create segmented line selections by specifying more than two coordinate, for example makeLine(25,34,44,19,69,30,71,56).
\end{indentCom}

\item Line 14: Call graph plotting function (Line 20 - 27), passing
\ilcom{totalprofile} array as an argument.

\item Line 15: call function to printout the profile array in the results window
(Lines 32 - 39).

\item Line 20 - 27: Function for plotting the intensity profile.
\item Line 21 : Use \ilcom{Array.getStatistics} function to know the minimum and
the maximum value of the array that was given as argument.
\item Line 22: Creates the window and axes of the plot. 
\item Line 23: Set the range for x and y axis using the results of line 21
\ilcom{min} and \ilcom{max}. 5\% of offset is added to both values for some
margins below and above.
\item Line 24: Sets the color of the plot. 
\item Line 25: Plot the profile. 
\item Line 26: Show the plot on the screen (lot is hidden until this show()
function).

\item Line 30 - 37: Function for outputting the profile array in the result
table. This function is exactly the same function you already used in the
previous chapter (code 20.5).

\end{itemize}
