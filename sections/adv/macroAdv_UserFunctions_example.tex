% This subsubsection can only be added with 
% conditionsloops/macroWhileApplication.tex
% using the code 14 in the above subsection (threshold adjustment for segmenting microtubule)
% some parts of the code are wrapped into user-defined-functions.
\subsubsection{Example of Creating Function}

Let's go back to the code 14, the automatic threshold adjusting macro. 

At the beginning of the code, we check if the active image if it is a stack. 
There is another check after that, to see if the image is adjusted with threshold level.

\lstinputlisting[linerange={1-10}]{code/code14.ijm}

We can make a function for checking stack (line 3 to 5) and another function that checks 
if the stack is adjusted with threshold level (from line 6 to 9) as below.

\lstinputlisting{code/code16_17functions.ijm}

Then the initial part of code 14 (line 3 to 9) can now be replaced with 
these two functions\footnote{For a complete coding of 14.1, 
getThreshold(lower, upper) should appear again in line 8 to get lower and upper threshold value 
of the reference image.}. 

\lstinputlisting[linerange={1-5}]{code/code14_1.ijm}

\begin{indentexercise}{1}
The following macro asks the user to input x and y coordinates of two points, 
calculate the distance between those points and prints out the distance. 
Modify the code so that the distance calculation is done in a separate function. 

\lstinputlisting{code/code18.ijm}

Note that function \ilcom{pow()} in the code is defined as
\begin{indentCom}
\textbf{pow}(base, exponent)\\
Returns the value of base raised to the power of exponent. 
\end{indentCom}
For example, \ilcom{pow(4, 2)} returns 16.
\end{indentexercise}